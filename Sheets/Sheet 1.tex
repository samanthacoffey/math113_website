\documentclass[12pt]{article}


%----------Packages----------
\usepackage{amsmath}
\usepackage{amssymb}
\usepackage{amsthm}
\usepackage{amsrefs}
\usepackage{dsfont}
\usepackage{mathrsfs}
\usepackage{stmaryrd}
\usepackage[all]{xy}
\usepackage[mathcal]{eucal}
\usepackage{verbatim}   %% includes comment environment
\usepackage{fullpage}   %% smaller margins
\usepackage{relsize}    %% can make symbols larger
%\usepackage{bm}

\usepackage{color} %% theorem colors
\definecolor{purple}{rgb}{0.5,0.20,0.7}
\definecolor{navy}{rgb}{0,0.3,0.7}
\definecolor{forrest}{rgb}{0,0.55,0.25}
\definecolor{strawberry}{rgb}{0.9,0.17,0.3}
\definecolor{orange}{rgb}{0.9,0.40,0.2}
\definecolor{pink}{rgb}{0.85,0.50,0.75}

%----------Commands----------

%%penalizes orphans
\clubpenalty=9999
\widowpenalty=9999





%% bold math capitals
\newcommand{\bA}{\mathbf{A}}
\newcommand{\bB}{\mathbf{B}}
\newcommand{\bC}{\mathbf{C}}
\newcommand{\bD}{\mathbf{D}}
\newcommand{\bE}{\mathbf{E}}
\newcommand{\bF}{\mathbf{F}}
\newcommand{\bG}{\mathbf{G}}
\newcommand{\bH}{\mathbf{H}}
\newcommand{\bI}{\mathbf{I}}
\newcommand{\bJ}{\mathbf{J}}
\newcommand{\bK}{\mathbf{K}}
\newcommand{\bL}{\mathbf{L}}
\newcommand{\bM}{\mathbf{M}}
\newcommand{\bN}{\mathbf{N}}
\newcommand{\bO}{\mathbf{O}}
\newcommand{\bP}{\mathbf{P}}
\newcommand{\bQ}{\mathbf{Q}}
\newcommand{\bR}{\mathbf{R}}
\newcommand{\bS}{\mathbf{S}}
\newcommand{\bT}{\mathbf{T}}
\newcommand{\bU}{\mathbf{U}}
\newcommand{\bV}{\mathbf{V}}
\newcommand{\bW}{\mathbf{W}}
\newcommand{\bX}{\mathbf{X}}
\newcommand{\bY}{\mathbf{Y}}
\newcommand{\bZ}{\mathbf{Z}}

%% blackboard bold math capitals
\newcommand{\bbA}{\mathbb{A}}
\newcommand{\bbB}{\mathbb{B}}
\newcommand{\bbC}{\mathbb{C}}
\newcommand{\bbD}{\mathbb{D}}
\newcommand{\bbE}{\mathbb{E}}
\newcommand{\bbF}{\mathbb{F}}
\newcommand{\bbG}{\mathbb{G}}
\newcommand{\bbH}{\mathbb{H}}
\newcommand{\bbI}{\mathbb{I}}
\newcommand{\bbJ}{\mathbb{J}}
\newcommand{\bbK}{\mathbb{K}}
\newcommand{\bbL}{\mathbb{L}}
\newcommand{\bbM}{\mathbb{M}}
\newcommand{\bbN}{\mathbb{N}}
\newcommand{\bbO}{\mathbb{O}}
\newcommand{\bbP}{\mathbb{P}}
\newcommand{\bbQ}{\mathbb{Q}}
\newcommand{\bbR}{\mathbb{R}}
\newcommand{\bbS}{\mathbb{S}}
\newcommand{\bbT}{\mathbb{T}}
\newcommand{\bbU}{\mathbb{U}}
\newcommand{\bbV}{\mathbb{V}}
\newcommand{\bbW}{\mathbb{W}}
\newcommand{\bbX}{\mathbb{X}}
\newcommand{\bbY}{\mathbb{Y}}
\newcommand{\bbZ}{\mathbb{Z}}

%% script math capitals
\newcommand{\sA}{\mathscr{A}}
\newcommand{\sB}{\mathscr{B}}
\newcommand{\sC}{\mathscr{C}}
\newcommand{\sD}{\mathscr{D}}
\newcommand{\sE}{\mathscr{E}}
\newcommand{\sF}{\mathscr{F}}
\newcommand{\sG}{\mathscr{G}}
\newcommand{\sH}{\mathscr{H}}
\newcommand{\sI}{\mathscr{I}}
\newcommand{\sJ}{\mathscr{J}}
\newcommand{\sK}{\mathscr{K}}
\newcommand{\sL}{\mathscr{L}}
\newcommand{\sM}{\mathscr{M}}
\newcommand{\sN}{\mathscr{N}}
\newcommand{\sO}{\mathscr{O}}
\newcommand{\sP}{\mathscr{P}}
\newcommand{\sQ}{\mathscr{Q}}
\newcommand{\sR}{\mathscr{R}}
\newcommand{\sS}{\mathscr{S}}
\newcommand{\sT}{\mathscr{T}}
\newcommand{\sU}{\mathscr{U}}
\newcommand{\sV}{\mathscr{V}}
\newcommand{\sW}{\mathscr{W}}
\newcommand{\sX}{\mathscr{X}}
\newcommand{\sY}{\mathscr{Y}}
\newcommand{\sZ}{\mathscr{Z}}

\newcommand{\calC}{\mathcal{C}}


\renewcommand{\phi}{\varphi}

\renewcommand{\emptyset}{\O}

\providecommand{\abs}[1]{\lvert #1 \rvert}
\providecommand{\norm}[1]{\lVert #1 \rVert}


\providecommand{\ar}{\rightarrow}
\providecommand{\arr}{\longrightarrow}

\renewcommand{\_}[1]{\underline{ #1 }}


\DeclareMathOperator{\ext}{ext}


\newcommand{\mathhuge}[1]{\mathlarger{\mathlarger{\mathlarger{{#1}}}}}


%----------Theorems----------

\newtheorem{theorem}{\color{navy}Theorem}[section]
\newtheorem{proposition}[theorem]{Proposition}
\newtheorem{lemma}[theorem]{\color{purple}Lemma}
\newtheorem{corollary}[theorem]{\color{pink}Corollary}


\newtheorem{axiom}{Axiom}


\theoremstyle{definition}
\newtheorem{definition}[theorem]{\color{forrest}Definition}
\newtheorem{nondefinition}[theorem]{\color{forrest}Non-Definition}
\newtheorem{exercise}[theorem]{\color{orange}Exercise}
\newtheorem{remark}[theorem]{Remark}
\newtheorem{warning}[theorem]{\color{strawberry}Warning}


\numberwithin{equation}{subsection}


%----------Title-------------
\title{Sheet 1: Introducing the Continuum}
\author{Jordan Paschke}
\date{}

\begin{document}

\begin{center}
{\large  MATH 113 -- Honors Calculus\\
$\bullet$ \quad \textbf{SHEET 1: INTRODUCING THE CONTINUUM \quad $\bullet$}}  \\ 
\vspace{.2in}  
Jordan Paschke\\September 2, 2015
\end{center}

\bigskip \bigskip

%%-----sheet number for theorem counter-----
\setcounter{section}{1}   

\noindent This sheet introduces \textbf{the continuum} $\calC$ through a series of axioms.

\vspace{0.2cm}

%%%%%%%%%%%%%%%%%%
%%---AXIOM 1----%%
%%%%%%%%%%%%%%%%%%
\fbox{\fbox{\parbox{3.4in}{
\begin{axiom}  The continuum is a nonempty set $\calC$.  
\end{axiom}}}}

\vspace{0.2cm}

\noindent We often refer to elements of $\calC$ as \emph{points}.


\begin{definition}  Let $X$ be a set.  An \emph{ordering} on the set $X$ is a subset of $X \times X$, denoted by $<$, with elements $(x,y) \in <$ written as $x < y$, satisfying the following properties:

\begin{itemize}
\item[(a)] For all $x, y \in X$ such that $x \neq y$, either $x < y$ or $y < x$.
\item[(b)] For all $x, y \in X$, if $x < y$ then $x \neq y$.
\item[(c)] For all $x, y, z \in X$, if $x < y$ and $y < z$ then $x < z$.
\end{itemize}
\end{definition}

%%%%%%%%%%%%%%%%%%
%%---AXIOM 2----%%
%%%%%%%%%%%%%%%%%%
\fbox{\fbox{\parbox{3.2in}{
\begin{axiom}  There exists an ordering $<$ on $\calC$.
\end{axiom}}}}

\begin{theorem}  If $x$ and $y$ are points of $\calC$, then $x < y$ and $y < x$ are not both true.
\end{theorem}

\begin{definition}  If $A \subset \calC$ is a subset of $\calC$, then a point $a \in A$ is a \emph{first} point of $A$ if, for every element $x \in A$, either $a < x$ or $a = x$.  Similarly, a point $b \in A$ is called a \emph{last} point of $A$ if, for every $x \in A$, either $x < b$ or $x = b$.
\end{definition}

\begin{lemma}  If $A$ is a nonempty, finite subset of $\calC$, then $A$ has a first and last point.
\end{lemma}

\begin{theorem}  Suppose that $A$ is a set of $n$ distinct points in $\calC$, or, in other words, $A \subset \calC$ has cardinality $n$.  Then symbols $a_1, \dotsc, a_n$ may be assigned to each point of $A$ so that $a_1 < a_2 < \dotsm < a_n$, i.e. $a_i < a_{i + 1}$ for $1 \leq i \leq n - 1$.
\end{theorem}

\begin{definition}  If $x, y, z \in \calC$ and both $x < y$ and $y < z$, then we say that $y$ is \emph{between} $x$ and $z$.
\end{definition}

\begin{corollary}  Of three distinct points, one must be between the other two.
\end{corollary}

%%%%%%%%%%%%%%%%%%
%%---AXIOM 3----%%
%%%%%%%%%%%%%%%%%%
\fbox{\fbox{\parbox{2.8in}{
\begin{axiom}  $\calC$ has no first or last point.
\end{axiom}}}}

\begin{definition} If $a < b$, then the set of all points between $a$ and $b$ is called a \emph{region}, denoted by $\_{ab}$.  
\end{definition}

\begin{warning}  One often sees the notation $(a, b)$ for regions.  We will reserve the notation $(a, b)$ for ordered pairs in a product $A \times B$.  These are very different things.  
\end{warning}

\begin{theorem} If $x$ is a point of $\calC$, then there exists a region $\_{ab}$ such that $x \in \_{ab}$.
\end{theorem}

We now come to one of the most important definitions of this course:

\begin{definition}
Let $A$ be a nonempty subset of $\calC$.  A point $p$ of $\calC$ is called a \emph{limit point} of $A$ if every region $R$ containing $p$ has nonempty intersection with $A \setminus \{p\}$.  Explicitly, this means:
\[
\text{for every region $R$ with $p \in R$, we have $R \cap (A \setminus \{p \}) \neq \varnothing$.}
\]
\end{definition}

Notice that we do not require that a limit point $p$ of $A$ be an element of $A$.

\begin{theorem} If $p$ is a limit point of $A$ and $A \subset B$, then $p$ is a limit point of $B$.
\end{theorem}



\begin{lemma} If $\_{ab}$ is a region in $\calC$, then:
\[
\calC = \{ x \mid x < a \} \cup \{a\} \cup \_{ab} \cup \{b \} \cup \{ x \mid b < x \}.
\]
\end{lemma}

\begin{definition} If $\_{ab}$ is a region in $\calC$, then $\calC \setminus \left(\{a\} \cup \_{ab} \cup \{b\}\right)$ is called the \emph{exterior} of $\_{ab}$ and is denoted by $\ext{ab}$.
\end{definition}

\begin{lemma}  If $\_{ab}$ is a region in $\calC$, then:
\[
\calC = \ext{ab} \cup \{a\} \cup \{b\} \cup \_{ab}.
\]
\end{lemma}

\begin{lemma}  No point in the exterior of a region is a limit point of that region.  No point of a region is a limit point of the exterior of that region.
\end{lemma}


\begin{theorem}  If two regions have a point $x$ in common, their intersection is a region containing $x$.
\end{theorem}

\begin{corollary}  If $n$ regions $R_1, \dotsc, R_n$ have a point $x$ in common, then their intersection $R_1 \cap \dotsm \cap R_n$ is a region containing $x$.
\end{corollary}

\begin{theorem}  Let $A, B \subset \calC$.  If $p$ is a limit point of $A \cup B$, then $p$ is a limit point of $A$ or $B$.
\end{theorem}

\begin{corollary}  Let $A_1, \dotsc, A_n$ be $n$ subsets of $\calC$.  If $p$ is a limit point of $A_1 \cup \dotsm \cup A_n$, then $p$ is a limit point of at least one of the sets $A_k$.
\end{corollary}

The converse is also true, so we have both directions:

\begin{corollary}
Let $A_1, \dotsc, A_n$ be $n$ subsets of $\calC$.  Then $p$ is a limit point of $A_1 \cup \dotsm \cup A_n$ if and only if $p$ is a limit point of at least one of the sets $A_k$.
\end{corollary}

\begin{definition}  Two sets $A$ and $B$ are \emph{disjoint} if $A \cap B = \varnothing$.  
\end{definition}

\begin{theorem}  If $p$ and $q$ are distinct points of $\calC$, then there exist disjoint regions $R$ and~$S$ containing $p$ and $q$, respectively.
\end{theorem}

\begin{corollary}  A subset of $\calC$ consisting of one point has no limit points.
\end{corollary}

\begin{theorem} A finite subset $A \subset \calC$ has no limit points.
\end{theorem}

\begin{corollary}  If $A \subset \calC$ is finite and $x \in A$, then there exists a region $R$ such that $A \cap R = \{ x \}$.
\end{corollary}

\begin{definition}  A set is \emph{infinite} if it is not finite.
\end{definition}

\begin{theorem}  If $p$ is a limit point of $A$ and $R$ is a region containing $p$, then the set $R \cap A$ is infinite.
\end{theorem}

\begin{exercise}  Find realizations of the continuum $\left(\calC, <\right)$.  That is, find examples of sets $\calC$ endowed with a relation $<$ satisfying all of the axioms (so far).  Are they the same?  What does ``the same'' mean here?
\end{exercise}



\end{document}