\documentclass[12pt]{article}


%----------Packages----------
\usepackage{amsmath}
\usepackage{amssymb}
\usepackage{amsthm}
\usepackage{amsrefs}
\usepackage{dsfont}
\usepackage{mathrsfs}
\usepackage{stmaryrd}
\usepackage[all]{xy}
\usepackage[mathcal]{eucal}
\usepackage{verbatim}  %%includes comment environment
\usepackage{fullpage}  %%smaller margins
%----------Commands----------

%%penalizes orphans
\clubpenalty=9999
\widowpenalty=9999





%% bold math capitals
\newcommand{\bA}{\mathbf{A}}
\newcommand{\bB}{\mathbf{B}}
\newcommand{\bC}{\mathbf{C}}
\newcommand{\bD}{\mathbf{D}}
\newcommand{\bE}{\mathbf{E}}
\newcommand{\bF}{\mathbf{F}}
\newcommand{\bG}{\mathbf{G}}
\newcommand{\bH}{\mathbf{H}}
\newcommand{\bI}{\mathbf{I}}
\newcommand{\bJ}{\mathbf{J}}
\newcommand{\bK}{\mathbf{K}}
\newcommand{\bL}{\mathbf{L}}
\newcommand{\bM}{\mathbf{M}}
\newcommand{\bN}{\mathbf{N}}
\newcommand{\bO}{\mathbf{O}}
\newcommand{\bP}{\mathbf{P}}
\newcommand{\bQ}{\mathbf{Q}}
\newcommand{\bR}{\mathbf{R}}
\newcommand{\bS}{\mathbf{S}}
\newcommand{\bT}{\mathbf{T}}
\newcommand{\bU}{\mathbf{U}}
\newcommand{\bV}{\mathbf{V}}
\newcommand{\bW}{\mathbf{W}}
\newcommand{\bX}{\mathbf{X}}
\newcommand{\bY}{\mathbf{Y}}
\newcommand{\bZ}{\mathbf{Z}}

%% blackboard bold math capitals
\newcommand{\bbA}{\mathbb{A}}
\newcommand{\bbB}{\mathbb{B}}
\newcommand{\bbC}{\mathbb{C}}
\newcommand{\bbD}{\mathbb{D}}
\newcommand{\bbE}{\mathbb{E}}
\newcommand{\bbF}{\mathbb{F}}
\newcommand{\bbG}{\mathbb{G}}
\newcommand{\bbH}{\mathbb{H}}
\newcommand{\bbI}{\mathbb{I}}
\newcommand{\bbJ}{\mathbb{J}}
\newcommand{\bbK}{\mathbb{K}}
\newcommand{\bbL}{\mathbb{L}}
\newcommand{\bbM}{\mathbb{M}}
\newcommand{\bbN}{\mathbb{N}}
\newcommand{\bbO}{\mathbb{O}}
\newcommand{\bbP}{\mathbb{P}}
\newcommand{\bbQ}{\mathbb{Q}}
\newcommand{\bbR}{\mathbb{R}}
\newcommand{\bbS}{\mathbb{S}}
\newcommand{\bbT}{\mathbb{T}}
\newcommand{\bbU}{\mathbb{U}}
\newcommand{\bbV}{\mathbb{V}}
\newcommand{\bbW}{\mathbb{W}}
\newcommand{\bbX}{\mathbb{X}}
\newcommand{\bbY}{\mathbb{Y}}
\newcommand{\bbZ}{\mathbb{Z}}

%% script math capitals
\newcommand{\sA}{\mathscr{A}}
\newcommand{\sB}{\mathscr{B}}
\newcommand{\sC}{\mathscr{C}}
\newcommand{\sD}{\mathscr{D}}
\newcommand{\sE}{\mathscr{E}}
\newcommand{\sF}{\mathscr{F}}
\newcommand{\sG}{\mathscr{G}}
\newcommand{\sH}{\mathscr{H}}
\newcommand{\sI}{\mathscr{I}}
\newcommand{\sJ}{\mathscr{J}}
\newcommand{\sK}{\mathscr{K}}
\newcommand{\sL}{\mathscr{L}}
\newcommand{\sM}{\mathscr{M}}
\newcommand{\sN}{\mathscr{N}}
\newcommand{\sO}{\mathscr{O}}
\newcommand{\sP}{\mathscr{P}}
\newcommand{\sQ}{\mathscr{Q}}
\newcommand{\sR}{\mathscr{R}}
\newcommand{\sS}{\mathscr{S}}
\newcommand{\sT}{\mathscr{T}}
\newcommand{\sU}{\mathscr{U}}
\newcommand{\sV}{\mathscr{V}}
\newcommand{\sW}{\mathscr{W}}
\newcommand{\sX}{\mathscr{X}}
\newcommand{\sY}{\mathscr{Y}}
\newcommand{\sZ}{\mathscr{Z}}



%\renewcommand{\hline}{\noindent\rule{15cm}{0.4pt}}

\renewcommand{\phi}{\varphi}

\renewcommand{\emptyset}{\O}

\providecommand{\abs}[1]{\lvert #1 \rvert}
\providecommand{\norm}[1]{\lVert #1 \rVert}


\providecommand{\x}{\times}




\providecommand{\ar}{\rightarrow}
\providecommand{\arr}{\longrightarrow}


\pagestyle{empty}



%----------Theorems----------

\newtheorem{theorem}{Theorem}[section]
\newtheorem{proposition}[theorem]{Proposition}
\newtheorem{lemma}[theorem]{Lemma}
\newtheorem{corollary}[theorem]{Corollary}
\newtheorem{definition}[theorem]{Definition}

\numberwithin{equation}{subsection}


%----------Title-------------
\title{Homework \#0}
\author{Jordan Paschke}
\date{August $28^{\text{th}}$, 2015}

\begin{document}

\maketitle

Here are the two logic puzzles we discussed as homework in today's class:

\begin{enumerate}
    \item Suppose that there is a small island village with 6 residents, where any discussion of eye color is strictly forbidden. There are no reflective surfaces of any kind allowed, and since the villagers cannot talk about their eyes, then they are unaware of how many different eye colors exist among them.

If a villager should happen to figure out their eye color, then they are honor-bound to exile themselves from the village, at midnight the day they make the discovery.

As an explorer, you stumble across this remote village, and immediately notice that 3 of them have blue eyes, and 3 of them have brown eyes (remember that the villagers don't know there are only two colors of eyes among them). Without knowledge of the village's customs, you accidentally exclaim, ``I see some of you have blue eyes", before the villagers all warn you about what you have said, and forbid you from speaking ever again.

\textbf{Question}: Now that the villagers are aware that at least \emph{someone} among them has blue eyes, how many people will be forced to leave the village? How many days does it take before everyone who is exiled has left?

\textbf{Hint}: The answer is not ``no one leaves", and this is not a trick question. Try thinking about the same problem with a smaller number of people. What if there are only 2 of each eye color? What about 1 of each? Does the puzzle change if the number of each color are different from each other?

\noindent\rule{15cm}{0.4pt}

    \item Suppose that 10 prisoners are all on death row, and sentenced to be executed tomorrow. The warden tells them that he will line them up in a row, all facing the person directly in front of them. He also tells them that he will put a random hat on each of their heads (either black or white), and should they guess the color of their hat correctly, he will set them free.

The warden explains that he will begin placing the hats on everyone's heads, starting from the first person in line (the person who cannot see any other prisoners); he will continue down the line until he reaches the last person (the one who can see the most other inmates).  He will then begin asking this person for their guess, and continue towards the front of the line, one-by-one. He does not give any information about how many of each color he will use.

While any prisoner can see the people (and hats) in front of him, he cannot see his own hat or anything behind him; he can however hear all of the answers given prior to his. If a prisoner says anything besides ``black" or ``white", he will be executed on the spot.

Using this knowledge, the prisoners (each of whom has perfect logic skills, and wishes to see his friends live) discuss and devise a plan to maximize the number of them who will go free.

\textbf{Question}: What is the maximum number of prisoners that are \emph{guaranteed} to survive? What is such a plan that leads to this outcome?

\textbf{Hint}: It is of course possible to save at least one prisoner -- there are many different strategies that the prisoners could use for this outcome (e.g. they could all agree to say the color of the hat on the person at the front of the line). It is not possible, however, to guarantee the survival of everyone (why?) Also, try thinking about the same problem, but with a smaller number of prisoners. What happens if there are only 2 prisoners? 3 prisoners? What about more than 10?
\end{enumerate}

\noindent\makebox[\linewidth]{\rule{\paperwidth}{0.4pt}}

\vspace{0.5cm}

Please write up a detailed solution to your assigned problem to be turned in at the end of class on Monday. Practice writing out your thoughts in a logical, and sequential manner, similar to the way we worked out the coins puzzle in class today. You may write the solution out by hand if you desire, or in {\LaTeX} or Word (any of these will work, since there shouldn't be any math symbols going on in either solution).\\


I encourage you all to work with the people from your table, and decide on how you want to present your solution to the rest of the class. We will take the beginning of class on Monday to have the two groups present their solutions to each other. Despite working on the solutions together, please try to write up everything in your own words when you turn it in.


\end{document}